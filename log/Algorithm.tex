\documentclass[export, 12pt, letterpaper]{ctexrep}

\usepackage{titling}
\usepackage{standalone}
\usepackage{fontenc}


\usepackage[x11names]{xcolor}
\definecolor{SaddleBrown}{rgb}{0.55, 0.27, 0.07}
\definecolor{IndianRed}{rgb}{0.80, 0.36, 0.36}
\definecolor{Firebrick}{rgb}{0.70, 0.13, 0.13}
\definecolor{DarkBlue}{rgb}{0, 0, 0.55}
\definecolor{Goldenrod}{rgb}{0.85, 0.65, 0.13}
\definecolor{Peru}{rgb}{0.85, 0.65, 0.13}




\usepackage{hyperref}
\hypersetup{
    colorlinks=true,
    linkcolor=DarkBlue,
    urlcolor=blue,
    filecolor=DarkBlue,
    linktoc=all
}
\setcounter{tocdepth}{1}


\usepackage{mathtools}
\usepackage{amsmath}
\usepackage{tikz}
\usepackage{verbatim}
\usetikzlibrary{calc}
\usetikzlibrary{shapes.geometric}

% put color to \boxed math command
\newcommand*{\boxcolor}{Peru}
\makeatletter
\renewcommand{\boxed}[1]{\textcolor{\boxcolor}{%
\tikz[baseline={([yshift=-1ex]current bounding box.center)}] \node [rectangle, minimum width=1ex,rounded corners,draw] {\normalcolor\m@th$\displaystyle#1$};}}
\makeatother


\usepackage{listings}

\definecolor{codegreen}{rgb}{0,0.6,0}
\definecolor{codegray}{rgb}{0.5,0.5,0.5}
\definecolor{codepurple}{rgb}{0.58,0,0.82}
\definecolor{backcolour}{rgb}{0.95,0.95,0.92}

\lstdefinestyle{mystyle}{
    backgroundcolor=\color{backcolour},
    commentstyle=\color{codegreen},
    keywordstyle=\color{magenta},
    numberstyle=\tiny\color{codegray},
    stringstyle=\color{codepurple},
    basicstyle=\ttfamily\footnotesize,
    breakatwhitespace=false,
    breaklines=true,
    captionpos=b,
    keepspaces=true,
    numbers=left,
    numbersep=5pt,
    showspaces=false,
    showstringspaces=false,
    showtabs=false,
    tabsize=2
}

\lstset{style=mystyle}


\usetikzlibrary{positioning}
\usetikzlibrary{petri} % LATEX and plain TEX
\usetikzlibrary[petri] % ConTEXt
\usepackage{pgfplots}
\pgfplotsset{compat=1.13}


\usepackage{framed}
\usepackage{quoting}

\colorlet{shadecolor}{LightSteelBlue1}
\usepackage{lipsum}
\newenvironment{shadedquotation}
 {\begin{shaded*}
  \quoting[leftmargin=5pt, rightmargin=5pt, vskip=0pt]
 }
 {\endquoting
 \end{shaded*}
}


\usepackage{lipsum} % for dummy text
\usepackage{enumitem}
\setlist{nosep} % or \setlist{noitemsep} to leave space around whole list


\usepackage{graphicx}
\usepackage{adjustbox}
\usepackage{caption}
\usepackage{float}

\pagestyle{plain}

        

\usepackage{booktabs, multirow} % for borders and merged ranges
\usepackage{soul}% for underlines
%\usepackage[table]{xcolor} % for cell colors
\usepackage{changepage,threeparttable} % for wide tables
\usepackage{longtable} % table for mutiple pages


\usepackage{ctex}
\usepackage{xeCJK}
\xeCJKsetup{CJKmath=true}


            \usepackage[
                left=1.0in,
                right=1.0in,
                top=1.5in,
                bottom=1.5in,
            ]{geometry}
        

\setlength{\droptitle}{-18em}
\title{\fontsize{40}{20} \textbf{\textcolor{red}{\kaishu Algorithm}}}
\author{\textcolor{red}{南方小智}}
\date{\textcolor{red}{\today}}



\begin{document}

\maketitle
\thispagestyle{empty}
\newpage

\setcounter{page}{1}
\tableofcontents
\newpage

%Please add the following packages if necessary:
%\usepackage{booktabs, multirow} % for borders and merged ranges
%\usepackage{soul}% for underlines
%\usepackage[table]{xcolor} % for cell colors
%\usepackage{changepage,threeparttable} % for wide tables
%If the table is too wide, replace \begin{table}[!htp]...\end{table} with
%\begin{adjustwidth}{-2.5 cm}{-2.5 cm}\centering\begin{threeparttable}[!htb]...\end{threeparttable}\end{adjustwidth}
%\begin{table}[!htp]\centering
%\begin{adjustwidth}{1.5 cm}{-0.5 cm}\centering\begin{threeparttable}[!htb]
%\begin{tabular}{lrrr}\toprule
%\begin{tabular}{p{0.1\textwidth}p{0.2\textwidth}p{0.65\textwidth}} \toprule
\begin{longtable}{p{0.2\textwidth} | p{0.25\textwidth} p{0.55\textwidth}} \toprule
\textbf{Table Name} &\textbf{Column Name} &\textbf{Desc/Type} \\\midrule
\multirow{6}{*}{DiceHistory} &dice1 &PostiveInt \\ \cline{2-3}
&dice2 &PostiveInt \\ \cline{2-3}
&player &Player (哪位玩家掷的骰子) \\ \cline{2-3}
&turn\_id &Int(第几轮掷的骰子) \\ \cline{2-3}
&(action\_id) &Int (代表这个操作是该场游戏的第几个action,用来重播游戏过程,用于Debug) \\ \cline{2-3}
&game &Game \\ \cline{2-3}
\hline
\multirow{10}{*}{RobberHistory} &x &Int (Robber所在位置的x坐标) \\ \cline{2-3}
&y &Int (Robber所在位置的y坐标) \\ \cline{2-3}
&player &Player \\ \cline{2-3}
&is\_knight &Bool(True代表是由Knight技能卡牌触发的Robber,False代表通过掷骰子到数字7触发的Robber) \\ \cline{2-3}
&is\_latest &Bool(是否最后一个Robber操作,可用于查询当前Robber位置,每个游戏只有一个true value) \\ \cline{2-3}
&victim &Player(受害者玩家) \\ \cline{2-3}
&(cardset\_movement) &CardsetMovement(抽牌转移情况) \\ \cline{2-3}
&turn\_id &Int(第几轮发生的Robber事件) \\ \cline{2-3}
&(action\_id) &Int (代表这个操作是该场游戏的第几个action,用来重播游戏过程,用于Debug) \\ \cline{2-3}
&game &Game \\ \cline{2-3}
\hline
\multirow{10}{*}{Cardset} &lumber &Int (default=0) \\ \cline{2-3}
&brick &Int (default=0) \\ \cline{2-3}
&wool &Int (default=0) \\ \cline{2-3}
&grain &Int (default=0) \\ \cline{2-3}
&ore &Int (default=0) \\ \cline{2-3}
&dev\_knight &Int (default=0) \\ \cline{2-3}
&dev\_one\_victory\_point &Int (default=0) \\ \cline{2-3}
&dev\_road\_building &Int (default=0) \\ \cline{2-3}
&dev\_monopoly &Int (default=0) \\ \cline{2-3}
&dev\_year\_of\_plenty &Int (default=0) \\ \cline{2-3}
\hline
\multirow{6}{*}{Player} &card\_set &CardSet \\ \cline{2-3}
&order &Int (取值范围是0~玩家数,代表该玩家是第几个开始行动的玩家。) \\ \cline{2-3}
&color &Chars (玩家的颜色) \\ \cline{2-3}
&knight number &(可以根据RobberHistory进行计算) \\ \cline{2-3}
&(user) &Int (用户id,外层的Portal系统负责用户的注册,有用户名,头像等信息。头像信息也可以放到每个Game里,也就是每个游戏可以随时设置不同头像) \\ \cline{2-3}
&game &Game \\ \cline{2-3}
\hline
\multirow{6}{*}{Construction} &type &Chars(可选House, Town, Road) \\ \cline{2-3}
&owner &Player(物件所属玩家) \\ \cline{2-3}
&x &Int (所在位置x坐标) \\ \cline{2-3}
&y &Int (所在位置y坐标) \\ \cline{2-3}
&z &Int (所在位置z坐标) \\ \cline{2-3}
&game &Game \\ \cline{2-3}
\hline
\multirow{5}{*}{Tile} &type &Chars(可选五种基本资源,Sea,Desert) \\ \cline{2-3}
&number &Int (每个地块上的数字,2~12) \\ \cline{2-3}
&x &Int (所在位置x坐标) \\ \cline{2-3}
&y &Int (所在位置y坐标) \\ \cline{2-3}
&game &Game \\ \cline{2-3}
\multirow{4}{*}{HarborSea} &type &Chars(可选五种基本资源,Any3) \\ \cline{2-3}
&x &Int (所在海洋位置x坐标) \\ \cline{2-3}
&y &Int (所在海洋位置y坐标) \\ \cline{2-3}
&game &Game \\ \cline{2-3}
\multirow{5}{*}{HarborLand} &x &Int (所在位置x坐标) \\ \cline{2-3}
&y &Int (所在位置x坐标) \\ \cline{2-3}
&z &Int (所在位置x坐标) \\ \cline{2-3}
&sea &HarborSea \\ \cline{2-3}
&game &Game \\ \cline{2-3}
\hline
\multirow{2}{*}{Bank} &cardset &Cardset \\ \cline{2-3}
&game &Game \\ \cline{2-3}
\hline
\multirow{6}{*}{Game} &map\_name &Chars(游戏用的地图模版名) \\ \cline{2-3}
&turn\_id &Int (当前是该场游戏的第几个回合,base=0,每个回合可能有多个action,前2N个回合为Settle阶段,N为玩家数) \\ \cline{2-3}
&status &Chars(游戏当前的阶段,包括 settle: 放房子阶段。 main:主游戏阶段。 end:游戏结束显示结果阶段。 ) \\ \cline{2-3}
&number\_of\_player &Int (总玩家数量,也可以从Player表计算) \\ \cline{2-3}
&(action\_id) &Int (当前是该场游戏的第几个action) \\ \cline{2-3}
&curr\_player &Player(当前玩家) \\ \cline{2-3}
\bottomrule
%\end{tabular}
\caption{数据库设计}\label{tab: }
\scriptsize
\end{longtable}
%\end{table}
%\end{threeparttable}\end{adjustwidth}


\chapter{排序}





\chapter{贪心}





\chapter{搜索}


\section{DFS深度优先搜索}


\section{BFS宽度优先搜索}


\section{迭代深搜}




\chapter{数据结构}


\section{树状树组}


\section{字典树}


\section{kd树}


\section{并查集}


\section{线段树}


\section{左偏树}


\section{点树}




\chapter{编译原理}


\section{逆波兰表达式}





\end{document}
